% package inclusion
\usepackage[usenames,dvipsnames,svgnames]{xcolor}
\usepackage{amsthm}
\usepackage{amsmath}
\usepackage{amssymb}
\usepackage{stmaryrd}
\usepackage{setspace}
\usepackage{amstext}
\usepackage[numbers,sort&compress]{natbib}
\usepackage{prettyref}
\usepackage{verbatim}
\usepackage[greek,english]{babel}
\usepackage{mdframed}
\usepackage{graphicx}
\usepackage{enumitem}
\usepackage{supertabular}
\usepackage{mathtools}
\usepackage{xstring}

% formatting macros
\newcommand{\keyword}[1]{\textsf{\textbf{#1}}}
\newcommand{\id}[1]{\textsf{\textsl{#1}}}
\newcommand{\tick}{\text{\textquoteright}}
\newcommand{\package}[1]{\textsf{#1}}
\newcommand{\ext}[1]{\texttt{#1}}
\newcommand{\flag}[1]{\texttt{#1}}

%% Import ``mathb'' font
\DeclareFontFamily{U}{mathb}{\hyphenchar\font45}
\DeclareFontShape{U}{mathb}{m}{n}{
      <5> <6> <7> <8> <9> <10> gen * mathb
      <10.95> mathb10 <12> <14.4> <17.28> <20.74> <24.88> mathb12
      }{}
\DeclareSymbolFont{mathb}{U}{mathb}{m}{n}
\DeclareFontSubstitution{U}{mathb}{m}{n}

%% just so I can get this symbol
\DeclareMathSymbol{\longrightsquigarrow}{3}{mathb}{"F9}

%% and I need \leftsquigarrow, as per
%% https://tex.stackexchange.com/questions/195025/how-to-get-the-opposite-direction-of-rightsquigarrow
\makeatletter
\providecommand{\leftsquigarrow}{%
  \mathrel{\mathpalette\reflect@squig\relax}%
}
\newcommand{\reflect@squig}[2]{%
  \reflectbox{$\m@th#1\rightsquigarrow$}%
}
\makeatother

\newcommand{\at}{@}
\newcommand{\pipe}{|}
\newcommand{\ok}{\ensuremath{\;\mathsf{ok}}}

\makeatletter
\newcommand{\raisemath}[1]{\mathpalette{\raisem@th{#1}}}
\newcommand{\raisem@th}[3]{\raisebox{#1}[0pt][0pt]{$#2#3$}}
\makeatother

\newcommand{\upi}{\mathmakebox[0pt][l]{\raisemath{-1.1\height}{\tilde{\phantom{\Pi}}}}\Pi}
\newcommand{\mpi}{\mathmakebox[0pt][l]{\raisemath{-1.15\height}{\bar{\phantom{\Pi}}}}\Pi}
\newcommand{\mupi}{\mathmakebox[0pt][l]{\raisemath{-1.3\height}{\tilde{\phantom{\Pi}}}}\mpi}

\input{ottdefns}
\newcommand{\rul}[1]{\textmd{\ottdrulename{#1}}}

\renewenvironment{ottfundefnblock}[3][]%
{\csname align*\endcsname}%
{\csname endalign*\endcsname}

\renewcommand{\ottfunclause}[2]{ #1 &= #2 \\}
\renewcommand{\ottkw}[1]{\ensuremath{\mathbf{#1}}}
